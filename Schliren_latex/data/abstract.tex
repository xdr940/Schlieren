\begin{abstract}


%我们提出了一个新的,基于开源游戏Minecraft 的虚拟数据集构建方法, 可以为视觉导航,深度估计等计算机视觉任务服务.
%该数据集通过插件程序, 提供一个渲染后的图像序列, 并带有时间对准的且与图像序列匹配的深度图, 表面法向图, 以及摄像机的 position and rotation.
%该数据集为高度可拓展, 用户不仅可以在社区中轻易取得大量场景文件省略去建模时间, 也可以根据需求自己在游戏里
%构建场景. 而且由于开源特性, 用户也可以根据自己的研究需要,编程实现生成其他ground-truth的功能插件
%与其他虚拟数据集不同的是, 我们提出数据集拥有大量玩家组成的社区, 能够以更低的成本去构建
%场景,获取工具,并且由于有大量光影渲染工具, 生成的虚拟数据集能够大大减少与真实数据之间的分布偏差.




We  present a novel synthetic dataset MinNav based on the sandbox game Minecraft. 
This dataset uses  several plug-in program to generate  rendered image sequences with time-aligned depth maps, surface normal maps and camera poses.
MinNav is a highly scalable dataset, users not only can easily obtain a large number of big-scale 3d scene files in the community, saving modeling time, but also can build specific scenes in the game. %according to different purpose. 
what's more, thanks for its open source feature, users can also develop modules to obtain other ground-truth for different purpose in research. 
Different from other synthetic datasets, our proposed dataset has a community of a large number of players, which can build 3d scenes and obtain tools at a lower cost, and because there are a large number of light and shadow rendering tools, the generated synthetic dataset can be greatly reduced Distribution deviation from real-world data.

\end{abstract}