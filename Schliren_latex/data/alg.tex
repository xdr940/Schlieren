\section{算法流程}
根据原理, 背景处流场密度为整个流场最低, 故将其映射为最低,设为0; 流场区域则根据公式\ref{eq:jetmap}映射, 综上所述,即可得算法\ref{ag:all}:

%\begin{eqnarray}
%    \quad &I_b(p) \rightarrow 0&\\
%    \quad &I_c = I_b + I_f&\\
%    \quad& I^*_c = \mathcal{F}(I_c)&\\
%    \quad&I^*_r(p) = \textbf{r} = [0,0,0]&
%\end{eqnarray}

\begin{algorithm}[h]  
    \caption{Schlieren算法}  
    \label{alg:Framwork}  
    \begin{algorithmic}[1]  
      \Require  
      $I$
      \Ensure  
     $I^*$
     \State $I_b = I\cdot M_b$
     \State $I_r = I\cdot M_r$
     \State $I_f = I\cdot M_f$
     \State $\forall p \in V_b,~~I_b(p) \leftarrow 0$
     \State $I^*_c = \mathcal{F}(I_b + I_f)$
     \State $\forall p \in V_r,~~I^*_r(p) = \textbf{r} = [0,0,0]$
     \State $I^* = I^*_c + I^*_r$
    \end{algorithmic}  
    \label{ag:all}
  \end{algorithm}
其中$I$为原灰度图, $I^*$为最后的渲染图(图.\ref{fig:first_show}右), $I^*_c$为映射后的区域,为RGB图;$r$为自定义的试件和管壁区域的颜色, 这里定义为黑色([0,0,0]); $I^*_r(p)$为Rigid 区域处理后的RGB图.